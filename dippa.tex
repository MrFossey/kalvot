\documentclass{beamer}
\usepackage[finnish]{babel}
\usepackage{beamerthemesplit} % new 
\usepackage[utf8]{inputenc}
\usepackage[T1]{fontenc}
\usepackage[scaled]{helvet}
\usepackage[round]{natbib}
\usepackage{tikz}
\usepackage{changepage}
\usetikzlibrary{positioning}
\usetikzlibrary{calc}
\usetikzlibrary{arrows}
\usetikzlibrary{decorations.pathmorphing,decorations.markings}
\usetikzlibrary{shapes}
\usetikzlibrary{patterns}
\usetikzlibrary{chains}
\usetikzlibrary{backgrounds, fit}


\begin{document}
\title{Diplomityösuunnitelma} 
\subtitle{IT CAN FELL ROCKETS or deloping an integration testing framework in large power sulutions supplier}
\author{Antti Heikkonen} 
\date{\today}

\frame{\titlepage} 

\frame{\frametitle{Sisältö}\tableofcontents} 

\section{Työn taustat} 

\frame{\frametitle{Työn tarkoitus} 

\begin{figure}
\centering
\begin{tikzpicture} [
   start chain=going below,        % General flow is top-to-bottom
    node distance=2mm and 20mm,    % Global setup of box spacing
    ]
% ------------------------------------------------- 
\tikzset{
  base/.style={draw, on chain, on grid, align=center, minimum height=2ex},
  rect/.style={base, rectangle, minimum height=2em, text width=3em},
  bus/.style={base, rectangle, minimum width=15em, text width=4em},
  line/.style={draw, thick, -latex'},
}
    \node [rect, xshift=0.75cm] (s1) {Service 1};
    \node [bus] (esb) {ESB};
  
    \node [rect, xshift=-1.00cm] (s21) {};
    \node [rect, xshift=0.13cm, yshift=0.75cm, fill=white] (s22) {};
    \node [rect, xshift=0.13cm, yshift=0.75cm, fill=white] (s23) {};
    \node [rect, xshift=0.13cm, yshift=0.75cm, fill=white, label=below:{System 2}] (s24) {Service};
    
    \node [rect, right=of s21] (s31) {};
    \node [rect, xshift=0.13cm, yshift=0.75cm, fill=white] (s32) {};
    \node [rect, xshift=0.13cm, yshift=0.75cm, fill=white] (s33) {};
    \node [rect, xshift=0.13cm, yshift=0.75cm, fill=white, label=below:{System 3}] (s34) {Service};
    
    \node [rect, right=of s31] (s41) {};
    \node [rect, xshift=0.13cm, yshift=0.75cm, fill=white] (s42) {};
    \node [rect, xshift=0.13cm, yshift=0.75cm, fill=white] (s43) {};
    \node [rect, xshift=0.13cm, yshift=0.75cm, fill=white, label=below:{System 4}] (s44) {Service};
  
    \path [line] (s1) -- (esb);
    \path [line] (s21) -- (esb);
    \path [line] (s31) -- (esb);
    \path [line] (s41) -- (esb);
    
\end{tikzpicture}
\caption{Service-oriented architecture} \label{fig:soa}
\end{figure}
Mitä tapahtuu ESB:n mustan laatikon sisällä?
}
% process issues: uniikit vaatimukset ja technical issues: miten käytännössä toteutetaan. viitekehys unohdetaan

\frame{\frametitle{Mikä framework?}
Frameworkilla tarkoitetaan teknistä viitekehystä. Se on IT-maailman prosessikartta, eikä ota kantaa käytettävään teknologiatuotteisiin, eikä toimi käyttäjien käyttöoppaana.
}

% Kysymystö vaikuttaa lavealta ja laajalta. Toisaalta se mukailee työn rakennetta: kappaleet vastaavat työn kysymyksiin. Haiskahtaa jälkeenpäin säädetyltä.
\frame{\frametitle{Tutkimuskysymykset}
Tutkimuskysymykset: \pause
\begin{itemize} %
\item Minkälaisian vaatimuksia integraatiotestausprosessilla? \pause
\item Miten framework pitäisi rakentaa yleiseten testaus- ja Wärtsilän omien vaatimusten sekä teknisen ympäristön valossa? \pause
\end{itemize}

}

% suurin ongelma että ei kuvata empiriistä toimenpidettä tarpeeksi tarkasti. mitä uutta luodaan ylläpitomallin tutkimisen ja referenssimallin luonnin ohella. dippa tuntuu enemmänkin selvitykseltä kuin tutkimukselta joka tuottaa uutta tietoa

\frame{\frametitle{Mutta!}
Työn tavoitetta ei kuvata.
}

\section{Tutkimuksen rakenne} 

\frame{\frametitle{Tutkimuksen rakenne}
\begin{figure}
  \includegraphics[width=6cm]{partanen_tutkimus.png}
    \caption{Partasen diplomityön rakenne}
    \label{fig:rakenne}
\end{figure}
}

\frame{\frametitle{Tutkimusmenetelmät}
Tutkimusmenetelminä action research ja konstruktiivinen tutkimus.
}

\section{Teoria} % Todella helppolukuista ja sujuvasti etenevää, mutta miksi juuri nämä asiat käsitellään.
\frame{\frametitle{Ohjelmistoylläpidon teoria}
Yleisten määritelmien ohella teoriaosuus esittelee
\begin{itemize}
\item Ylläpitoprosessien typologian, % menetelmät
\item Palvelunäkökulman ohjelmistoylläpitoon, % suhteet
\item Elinkaarinäkökulman ohjelmistoihin, % ympäristö
\item Neljä ylläpidon viitekehystä ja näiden vertailun, % luettele jos jotakuta kiinnostaa
\end{itemize}
joista kootaan \bf{referenssimalli}.
}
% referenssimallit käydään läpi aika tarkasti, mutta näiden valitsemiselle esitettiin kriteerit/hyvä scope. mallit kuvattu hyvin
% vertailussakin kriteeristö, joka esitellään, mutta vedetään vähän hatusta . plussaa oli että näitä käytettiin referenssimallin kehittämisessä eivätkä nämä jääneet näin irrallisiksi
% referenssimallin rakentamista olisi voinut korostaa alussa. tutkimusongelma jää edelleen epämääräiseksi selvittelyksi...
% referenssimalli käydään läpi todella tarkkaan prosessi prosessilta. tässä oma dippa eroaa aika paljon, vaikka karkea rakenne on sama.
% rakenneharjoitus
% lopussa teksti vähän heikkenee (kirjoitettu viimeiseksi? dipan versio ei lopullinen, Huom!)

\frame{\frametitle{Referenssimalli}
\begin{figure}
  \includegraphics[width=6cm]{partanen_malli.png}
    \caption{Ohjelmistoylläpidon referenssimalli}
    \label{fig:malli}
\end{figure}
}

\section{Empiria}
\frame{\frametitle{Ylläpitomentelmät Ixonosilla}
Haastatteluihin perustuva selvitys pureutui:
\begin{itemize}
\item Ongelmakohtiin valituissa projekteissa,
\item Ylläpitäjien kokemuksiin,
\item Asiakasedustajan kokemuksiin
\end{itemize}
ja tiivisti tulokset parannusehdotuksiin ja \bf{ylläpidon tarkistuslistaan} \it{(maintainability checklist)}.
}
%latistavaa kuulla että ylläpito enimmäkseen inaktiivista
% neljännen tason väliostikoita eikä kunnon johdatusta sisältöön, ongelmakohdat vain listataan

\frame{\frametitle{Uuden mallin arviointi -workshop}

Referenssimalli esitettiin ylläpitäjille validointiworkshopissa. 
Tulokset:
\begin{itemize}
\item Ehdotuksia uusista prosesseista
\item Päivitetty refenssimalli % lisätty prosesseja joita ei aiemmin ollut
\item Referenssimallista johdettu asiakaslähetöinen malli
\end{itemize}
}

\section{Yhteenveto}

\frame{\frametitle{Tutkimuksen arviointi}
Tutkimus täyttää hyvän tutkimuksen kriteerit, mutta referenssimalli kehitetty kohdeyrityksen tarpeisiin, eikä sitä ole sovellettu ja arvioitu päivittäisessä työssä.
}

\frame{\frametitle{Kokonaisvaikutelma Partasen dipasta}
Hyvää
\begin{itemize}
\item Nopeasti asiaan
\item Selkeä perusrakenne ja jouheva eteneminen: teoriatutkimus, synteesi, koe, johtopäätökset
\item Perusteellinen ja analyyttinen käsittely
\end{itemize}
\\[0.1in]
Huonoa
\begin{itemize}
\item Epäselvät tavoitteet
\item Epäselvä johdattelu empiriaan ja palautus takaisin teoriatasolle
\end{itemize}
}

\end{document}