\documentclass{beamer}
\usepackage[finnish]{babel}
\usepackage{beamerthemesplit} % new 
\usepackage[utf8]{inputenc}
\usepackage[T1]{fontenc}
\usepackage[scaled]{helvet}
\usepackage[round]{natbib}

\begin{document}
\title{Diplomityövertailu} 
\subtitle{Teemu Partanen: Prerequisites for successful software maintenance and a reference model of the maintenenace processes}
\author{Antti Heikkonen} 
\date{\today}

\frame{\titlepage} 

\frame{\frametitle{Sisältö}\tableofcontents} 

\section{Työn taustat} 

\frame{\frametitle{Työn tarkoitus} 
Työssä arvioitiin ohjelmistojen huolto/ylläpitöprosesseja (\emph{software maintenenance processes}) Ixonos Technology Consulting Ltd:lle, joka tarjoaa ylläpitoratkaisuja Ixonosin kehittämiin ohjelmiin. 
\\[0.2in]
Miksi? % dipassa: maksaa (hallinta, kehitys), ei huomiotu tutkimuksessa tai organisaatioissa
\begin{itemize} 
\item Toistuva prosessi, jota kannattaa optimoida
\item Elinkaarinäkökulma: kustannuksia ja (lisä)myyntiä myös toimituksen jälkeen
\end{itemize}
}
% process issues: uniikit vaatimukset ja technical issues: miten käytänössä toteutetaan. viitekehys unohdetaan

% Kysymystö vaikuttaa lavealta ja laajalta. Toisaalta se mukailee työn rakennetta: kappaleet vastaavat työn kysymyksiin. Haiskahtaa jälkeenpäin säädetyltä.
\frame{\frametitle{Tutkimuskysymykset}
Tutkimuskysymykset: \pause
\begin{itemize} %
\item Mitkä ovat ohjelmistoylläpidon edellyttämät prosessit? \pause
\item Mitä olemassa olevia ylläpitomalleja on ja miten ne sopivat kohderitykseen? \pause
\item Mitä prosesseja kohdeyritys edellyttää ylläpitomallilta? \pause
\end{itemize}
Lisäksi
\begin{itemize}
\item Mitä ongelmia on nykyisessä ylläpitomallissa? \pause
\item Miten ylläpitoa voidaan fasilitoida kehitysvaiheessa ongelmien välttämiseksi? % yhdistää teoriaa ja empiriaa
\end{itemize}
}

% suurin ongelma että ei kuvata empiriivistä toimenpidettä tarpeeksi tarkasti. mitä uutta luodaan ylläpitomallin tutkimisen ja referenssimallin luonnin ohella. dippa tuntuu enemmänkin selvitykseltä kuin tutkimukselta joka tuottaa uutta tietoa

\section{Tutkimuksen rakenne} 

\frame{\frametitle{Tutkimuksen rakenne}
\begin{figure}
  \includegraphics[width=6cm]{partanen_tutkimus.png}
    \caption{Partasen diplomityön rakenne}
    \label{fig:rakenne}
\end{figure}
}

\frame{\frametitle{Tutkimusmenetelmät}
Tutkimusmenetelminä action research ja konstruktiivinen tutkimus.
}

\section{Teoria} % Todella helppolukuista ja sujuvasti etenevää, mutta miksi juuri nämä asiat käsitellään.
\frame{\frametitle{Ohjelmistoylläpidon teoria}
Yleisten määritelmien ohella teoriaosuus esittelee
\begin{itemize}
\item Ylläpitoprosessien typologian, % mentelmät
\item Palvelunäkökulman ohjelmistoylläpitöön, % suhteet
\item Elinkaarinäkökulman ohjelmistoihin, % ympäristö
\item Neljä ylläpidon viitekehystä ja näiden vertailun, % luettele jos jotakuta kiinnostaa
\end{itemize}
joista kootaan \bf{referenssimalli}.
}
% referenssimallit käydään läpi aika tarkasti, mutta näiden valitsemiselle esitettiin kriteerit/hyvä scope. mallit kuvattu hyvin
% vertailussakin kriteeristö, joka esitellään, mutta vedetään vähän hatusta . plussaa oli että näitä käytettiin referenssimallin kehittämisessä eivätkä nämä jääneet näin irrallisiksi
% referenssimallin rakentamista olisi voinut korostaa alussa. tutkimusongelma jää edelleen epämääräiseksi selvittelyksi...
% referenssimalli käydään läpi todella tarkkaan prosessi prosessilta. tässä oma dippa eroaa aika paljon, vaikka karkea rakenne on sama
% rakenneharjoitus
% lopussa teksti vähän heikkenee (kirjotettu viimeiseksi? dipan versio ei lopullinen, Huom!)

\section{Empiria}
\frame{\frametitle{Empiria}
Selvitys nyksisistä ylläpitokäytännöistä kohdeyrityksessä
}

\section{Yhteenveto}
\frame{\frametitle{Kokonaisvaikutelma Partasen dipasta}
Työssä mentiin kiitettävän nopeasti asiaan. Rakenne oli selkeä ja eteni aihepiirin yleisestä esittelystä malleihin, ja näiden analyysiin ja lopuksi synteesiin uudesta toimintamallista. Kaksi viimesitä kappaletta empiirisestä tutkimuksesta olisi voinut jakaa tai pohjustaa paremmin. Kaartaa takaisin teoriaan joutuu etsimään. Voin kuitenkin hyödyntää tutkimuksen rakennetta omassa työssäni.
}

\end{document}