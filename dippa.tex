% This text is proprietary.
% It's a part of presentation made by myself.
% It may not used commercial.
% The noncommercial use such as private and study is free
% Sep. 2005 
% Author: Sascha Frank 
% University Freiburg 
% www.informatik.uni-freiburg.de/~frank/
% additional use of \usepackage{beamerthemesplit}
\documentclass{beamer}
\usepackage[finnish,swedish,english]{babel}
\usepackage{beamerthemesplit} % new 
\usepackage[utf8]{inputenc}
\usepackage[T1]{fontenc}
\usepackage[scaled]{helvet}

\begin{document}
\title{Diplomityövertailu} 
\subtitle{Teemu Partanen: Prerequisites for successful software maintenance and a reference model of the maintenenace processes}
\author{Antti Heikkonen} 
\date{\today} 

\frame{\titlepage} 

\frame{\frametitle{Sisältö}\tableofcontents} 

\section{Työn taustat} 

\frame{\frametitle{Työn tarkoitus} 
Työssä arvioitiin ohjelmistojen huolto/ylläpitöprosesseja (\emph{software maintenenance processes}) ohjelmistoyritys Ixonosille. 
\\
Miksi?
\begin{itemize}
\item Toistuva prosessi, jota kannattaa optimoida
\item Elinkaarinäkökulma: kustannuksia ja (lisä)myyntiä myös toimituksen jälkeen
\end{itemize}

}

\frame{\frametitle{Tutkimuskysymykset}
Tutkimuskysymykset: \pause
\begin{itemize}
\item Mitkä ovat ohjelmistoylläpidon edellyttämät prosessit? \pause
\item Mitä olemassa olevia ylläpitomalleja on ja miten ne sopivat kohderitykseen? \pause
\item Mitä prosesseja kohdeyritys edellyttää ylläpitomallilta? \pause
\end{itemize}
Lisäksi
\begin{itemize}
\item Mitä ongelmia on nykyisessä ylläpitomallissa? \pause
\item Miten ylläpitoa voidaan fasilitoida kehitysvaiheessa ongelmien välttämiseksi?
\end{itemize}
}

% tutkimuskysymyksiä on liikaa ja scope liian lavea

\section{Tutkimuksen rakenne} 

\frame{\frametitle{Tutkimuksen rakenne}
Tähän kuva
}

\frame{\frametitle{Huomioita tutkimuksen rakenteesta}
Työssä mentiin kiitettävän nopeasti asiaan. Rakenne oli selkeä ja eteni aihepiirin yleisestä esittelystä malleihin, ja näiden analyysiin ja lopuksi synteesiin uudesta toimintamallista. Kaksi viimesitä kappaletta empiirisestä tutkimuksesta olisi voinut jakaa tai pohjustaa paremmin. Kaartaa takaisin teoriaan joutuu etsimään. Voin kuitenkin hyödyntää tutkimuksen rakennetta omassa työssäni.
}

\section{Section no.3} 
\subsection{Tables}
\frame{\frametitle{Tables}
\begin{tabular}{|c|c|c|}
\hline
\textbf{Date} & \textbf{Instructor} & \textbf{Title} \\
\hline
WS 04/05 & Sascha Frank & First steps with  \LaTeX  \\
\hline
SS 05 & Sascha Frank & \LaTeX \ Course serial \\
\hline
\end{tabular}}


\frame{\frametitle{Tables with pause}
\begin{tabular}{c c c}
A & B & C \\ 
\pause 
1 & 2 & 3 \\  
\pause 
A & B & C \\ 
\end{tabular} }


\section{Section no. 4}
\subsection{blocs}
\frame{\frametitle{blocs}

\begin{block}{title of the bloc}
bloc text
\end{block}

\begin{exampleblock}{title of the bloc}
bloc text
\end{exampleblock}


\begin{alertblock}{title of the bloc}
bloc text
\end{alertblock}
}
\end{document}