\documentclass{beamer}
\usepackage[finnish]{babel}
\usepackage{beamerthemesplit} % new 
\usepackage[utf8]{inputenc}
\usepackage[T1]{fontenc}
\usepackage[scaled]{helvet}
\usepackage[round]{natbib}
\usepackage{tikz}
\usepackage{changepage}
\usetikzlibrary{positioning}
\usetikzlibrary{calc}
\usetikzlibrary{arrows}
\usetikzlibrary{decorations.pathmorphing,decorations.markings}
\usetikzlibrary{shapes}
\usetikzlibrary{patterns}
\usetikzlibrary{chains}
\usetikzlibrary{backgrounds, fit}


\begin{document}
\title{IT CAN FELL ROCKETS} 
\subtitle{or deloping an integration testing framework in large power solutions supplier}
\author{Antti Heikkonen} 
\date{\today}

\frame{\titlepage} 

\frame{\frametitle{Sisältö}\tableofcontents} 

\section{Työn taustat} 

\frame{\frametitle{Työn tavoite}
Työn tavoite on kehittää framework-prototyyppi järjestelmien väliseen integraatiotestaukseen.
}

% Wärtsilä’s business case is to save development and testing costs by automating test execution and test case creation where possible. Hypothetically, at least, standardizing the testing process, documenting it clearly, and communicating it to related parties contributes to efficiency, too, though only the process design is in the scope of the thesis.

\frame{\frametitle{Työn motivaatio} 

\begin{figure}
\centering
\begin{tikzpicture} [
   start chain=going below,        % General flow is top-to-bottom
    node distance=2mm and 20mm,    % Global setup of box spacing
    ]
% ------------------------------------------------- 
\tikzset{
  base/.style={draw, on chain, on grid, align=center, minimum height=2ex},
  rect/.style={base, rectangle, minimum height=2em, text width=4em},
  bus/.style={base, rectangle, minimum width=15em, text width=4em},
  line/.style={draw, thick, -latex'},
}
    \node [rect, xshift=0.75cm] (s1) {Service 1};
    \node [bus] (esb) {ESB};
  
    \node [rect, xshift=-1.00cm] (s21) {};
    \node [rect, xshift=0.13cm, yshift=0.75cm, fill=white] (s22) {};
    \node [rect, xshift=0.13cm, yshift=0.75cm, fill=white] (s23) {};
    \node [rect, xshift=0.13cm, yshift=0.75cm, fill=white, label=below:{System 2}] (s24) {Service};
    
    \node [rect, right=of s21] (s31) {};
    \node [rect, xshift=0.13cm, yshift=0.75cm, fill=white] (s32) {};
    \node [rect, xshift=0.13cm, yshift=0.75cm, fill=white] (s33) {};
    \node [rect, xshift=0.13cm, yshift=0.75cm, fill=white, label=below:{System 3}] (s34) {Service};
    
    \node [rect, right=of s31] (s41) {};
    \node [rect, xshift=0.13cm, yshift=0.75cm, fill=white] (s42) {};
    \node [rect, xshift=0.13cm, yshift=0.75cm, fill=white] (s43) {};
    \node [rect, xshift=0.13cm, yshift=0.75cm, fill=white, label=below:{System 4}] (s44) {Service};
  
    \path [line] (s1) -- (esb);
    \path [line] (s21) -- (esb);
    \path [line] (s31) -- (esb);
    \path [line] (s41) -- (esb);
    
\end{tikzpicture}
\caption{Wärtsilän SOA tietojärjestelmäarkkitehtuuri} \label{fig:soa}
\end{figure}
Mitä tapahtuu ESB:n (integraation) mustan laatikon sisällä?
}

% Wärtsilä is a Finnish technology company providing lifecycle power solutions for the marine and energy markets . As is the standard today, Wärtsilä’s operations rely on a number of IT systems under pressure to adapt to changing business requirements and contextual constraints. Testing is necessary to ensure quality and integrate new external systems, but is also recognized as an expensive and time-consuming endeavor – a focal point for improvement.

\frame{\frametitle{Mikä framework?}
Tässä yhteydessä frameworkilla tarkoitetaan teknistä viitekehystä. Se on IT-maailman prosessikartta.
% Pureudutaan taskeihin ja suoritusjärjestykseen
\\[0.1in]
Pèzze et al: "a circuit board with empty slots for components"
\\[0.1in]
% Diplomityö on ohje miekan takomiselle, ei sen käyttämiselle.
}

\frame{\frametitle{Tutkimuskysymykset}
Tutkimuskysymykset:
\begin{itemize} %
\item Minkälaisia vaatimuksia integraatiotestausprosessilla on?
\item Miten framework pitäisi rakentaa Wärtsilän omien ja yleisten testausvaatimusten, sekä teknisen ympäristön valossa?
\end{itemize}
}

% The third empirical part consists of a proof of concept and initial results. The organization scale implementation is not in the scope of the thesis. Issues that arise during the third part will be discussed. The third research question also includes evaluation how the end solution meets previously defined criteria. 

\section{Tutkimuksen rakenne}
% Taustat, frameworkit, framework osat, uusi framework, koe, tulokset, johtopäätökset ja discussion.\begin{figure}
\frame{\frametitle{Tutkimuksen rakenne}

\begin{figure}
\centering
\begin{tikzpicture} [
   start chain=going below,        % General flow is top-to-bottom
    node distance=4mm and 20mm,    % Global setup of box spacing
    ]
% ------------------------------------------------- 
\tikzset{
  base/.style={draw, on chain, on grid, align=center, minimum height=2ex},
  rect/.style={base, minimum height=2em, rectangle, text width=8em},
  bus/.style={base, rectangle, minimum width=15em, text width=4em},
  line/.style={draw, thick, -latex'},
}
    
    \node [rect, yshift=1.0cm] (p1) {Testaus-frameworkit (Teoria)};
    \node [rect] (p2) {Framework-komponentit (Analyysi)};
    \node [rect] (p3) {Uusi framework (Synteesi)};
    \node [rect, right=of p1, xshift=2.0cm] (p4) {Proof of Concept (Empiria)};
    \node [rect, right=of p2, xshift=2.0cm] (p5) {Tulokset};
    \node [rect, right=of p3, xshift=2.0cm] (p6) {Johtopäätökset};
    \node [rect, left=of p3, xshift=-2.0cm] (p0) {Taustat (Vaatimukset};
  
    \path [line] (p0) -- (p3);
    \path [line] (p1) -- (p2);
    \path [line] (p2) -- (p3);
    \path [line] (p3) -- (p4);
    \path [line] (p4) -- (p5);
    \path [line] (p5) -- (p6);
    
\end{tikzpicture}
\caption{Tutkimuksen rakenne} \label{fig:rakenne}
\end{figure}

}
\section{Tutkimusmenetelmät}
\frame{\frametitle{Analyysin viitekehys}
\begin{figure}[H]
\centering
\begin{tikzpicture} [ 
   start chain=going below,         % General flow is top-to-bottom
    node distance=10mm and 50mm,    % Global setup of box spacing
    thick,scale=0.6, every node/.style={transform shape}
    ]
% ------------------------------------------------- 
\tikzset{
  base/.style={draw, on chain, on grid, align=center, minimum height=4ex},
  rect/.style={base, rectangle, minimum height=2em, text width=6em},
  line/.style={draw, thick, -latex'},
  dots/.style={draw, dotted, -latex'}
}

    \draw[semithick] (-5,0) arc (180:0:2 and 0.5);                  % cone 1: top ellipse top arc
    \draw[dashed,color=gray] (-3.5,-5) arc (180:0:0.5 and 0.25);    % cone 1: bottom ellipse top arc
    \draw[semithick] (-3.5,-5) arc (-180:0:0.5 and 0.25);           % cone 1: bottom ellipse bottom arc
    \draw[semithick] (-5,0) arc (-180:0:2 and 0.5);                 % cone 1: top ellipse bottom arc
    \draw[semithick] (-5,0) -- (-3.50,-5);                          % cone 1: left side
    \draw[semithick] (-1,0) -- (-2.50,-5);                          % cone 1: right side
    
    \draw[semithick] (0,0) arc (180:0:2 and 0.5);                   % cone 2: top ellipse top arc
    \draw[dashed, color=gray] (1.5,-5) arc (180:0:0.5 and 0.25);    % cone 2: bottom ellipse top arc
    \draw[semithick] (1.5,-5) arc (-180:0:0.5 and 0.25);            % cone 2: bottom ellipse bottom arc
    \draw[semithick] (0,0) arc (-180:0:2 and 0.5);                  % cone 2: top ellipse bottom arc
    \draw[semithick] (0,0) -- (1.5,-5);                             % cone 2: left side
    \draw[semithick] (4,0) -- (2.5,-5);                             % cone 2: right side

    \draw[semithick] (5,0) arc (180:0:2 and 0.5);                   % cone 3: top ellipse top arc
    \draw[dashed, color=gray] (6.5,-5) arc (180:0:0.5 and 0.25);    % cone 3: bottom ellipse top arc
    \draw[semithick] (6.5,-5) arc (-180:0:0.5 and 0.25);            % cone 3: bottom ellipse bottom arc
    \draw[semithick] (5,0) arc (-180:0:2 and 0.5);                  % cone 3: top ellipse bottom arc
    \draw[semithick] (5,0) -- (6.5,-5);                             % cone 3: left side
    \draw[semithick] (9,0) -- (7.5,-5);                             % cone 3: right side

    \node [rect, xshift=-3.0cm, yshift=1.5cm, fill=white] (stg) {Strategy};
    \node [rect, yshift=0.25cm, fill=white] (req) {Requirements \& input};
    \node [rect, yshift=-0.75cm, fill=white] (prc) {Process};
    \node [rect, yshift=-1.4cm, fill=white] (rlt) {Results};
    
    \node [rect, right=of stg, fill=white] (fmw) {Framework};
    \node [rect, right=of req, fill=white] (tcg) {Test generation};
    \node [rect, right=of prc, fill=white] (run) {Running tests};
    \node [rect, right=of rlt, fill=white] (mon) {Monitoring/ verification};
    
    \node [rect, right=of fmw, fill=white] (art) {Objects};
    \node [rect, right=of tcg, fill=white] (tcs) {Test cases};
    \node [rect, right=of run, fill=white] (exe) {Execution engine};
    \node [rect, right=of mon, fill=white] (out) {Testing oracle};
    
  %  \node [rect, below=of mon, yshift=-1.0cm, fill=white] (pit) {Pitfalls};
     
    \path [line] (stg) -- (fmw);
    \path [line] (fmw) -- (stg);
    \path [line] (fmw) -- (art);
    \path [line] (art) -- (fmw);
    
    \path [line] (stg) -- (req);
    \path [line] (req) -- (prc);
    \path [line] (prc) -- (rlt);
    
    \path [line] (art) -- (tcs);
    \path [line] (tcs) -- (exe);
    \path [line] (exe) -- (out);
    
    \path [line] (fmw) -- (tcg);
    \path [line] (tcg) -- (run);
    \path [line] (run) -- (mon);
    
    \path [dots] (run) -- (prc);
    \path [dots] (prc) -- (run);
    \path [dots] (run) -- (exe);
    \path [dots] (exe) -- (run);
    
    \path [dots] (req) -- (tcg);
    \path [dots] (tcg) -- (req);
    \path [dots] (tcg) -- (tcs);
    \path [dots] (tcs) -- (tcg);
    
    \path [dots] (rlt) -- (mon);
    \path [dots] (mon) -- (rlt);
    \path [dots] (mon) -- (out);
    \path [dots] (out) -- (mon);
 
\end{tikzpicture}
\caption{Integraatiotestaustasot} \label{fig:funnel}
\end{figure}
}

\frame{\frametitle{Konstruktiivinen tutkimusmenetelmä}
Tarkoitus on tuottaa uusia hypoteeseja ja teorioita ja testata näitä käytännössä. 
\\[0.1in]
Konstruktiivinen tutkmus painottaa käytännönläheisyyttä ja sovellettavuutta ottaen huomioon laatu-, kustannus- ja ajatasaisuustekijät.
}

\section{Aikataulu}
\frame{\frametitle{Aikataulu}
\begin{figure}[H]
  \begin{center}
    \includegraphics[width=5cm]{schedule.png}
    \caption{Dipan aikataulu}
    \label{fig:aikataulu}
  \end{center}
\end{figure}
}

\frame{\frametitle{Aikataulu2}
Abdullah, Khalil; Kimble, Jim, and White, Lee. Correcting for unreliable regression integration testing. In Software Maintenance, 1995. Proceedings.,International Conference on, pages 232-241. IEEE, 1995.
\\[0.1in]
Benz, Sebastian. Combining test case generation for component and integration testing. In Proceedings of the 3rd international workshop on Advances in model-based testing, pages 23-33. ACM, 2007.
\\[0.1in]
Bhuyan, Prachet; Kashyap, Chandra Prakash, and Mohapatra, Durga Prasad. A survey of regression testing in SOA. International Journal of Computer Applications, 44(19):22{25, April 2012. Published by Foundation of Computer Science, New York, USA.
\\[0.1in]
Duvall, Paul M; Matyas, Steve, and Glover, Andrew. Continuous integration:
improving software quality and reducing risk. Addison-Wesley Professional,
2007.
\\[0.1in]
Fewster, Mark and Graham, Dorothy. Software test automation: effective use of
test execution tools. ACM Press/Addison-Wesley Publishing Co., 1999.
\\[0.1in]
Huang, He Yuan; Liu, He Hui; Li, Zhong Jie, and Zhu, Jun. Surrogate: A simulation apparatus for continuous integration testing in service oriented architecture. In Services Computing, 2008. SCC'08. IEEE International Conference on, volume 2, pages 223-{230. IEEE, 2008.
\\[0.1in]
Jenkins, Nick. A software testing primer. An Introduction to Software testing, e-book, 2008.
\\[0.1in]
Myers, Glenford J. Software reliability: principles and practices. Wiley, 1976.
Pezzè, Mauro and Young, Michal. Software testing and analysis: process, principles, and techniques. John Wiley & Sons Inc, 2008.
\\[0.1in]
Laukkanen, Pekka. Data-driven and keyword-driven test automation frameworks. Master’s thesis, Helsinki University of Technology, 2006.
\\[0.1in]
Leung, Hareton KN and White, Lee. A study of integration testing and software regression at the integration level. In Software Maintenance, 1990.,Proceedings., Conference on, pages 290-301. IEEE, 1990.
\\[0.1in]
Linnenkugel, Ursula and Mullerburg, Monika. Test data selection criteria for (software) integration testing. In Systems Integration, 1990. Systems Integration'90., Proceedings of the First International Conference on, pages 709-717. IEEE, 1990.
\\[0.1in]
Rehman, Jaffar-ur; Jabeen, Fakhra; Bertolino, Antonia; Polini, Andrea, and
others, . Testing software components for integration: a survey of issues
and techniques. Software Testing, Verification and Reliability, 17(2):95–133,
2007.
}

\begin{comment}

\end{comment}


\end{document}