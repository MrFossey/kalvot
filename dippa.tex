\documentclass{beamer}
\usepackage[finnish]{babel}
\usepackage{beamerthemesplit} % new 
\usepackage[utf8]{inputenc}
\usepackage[T1]{fontenc}
\usepackage[scaled]{helvet}
\usepackage[round]{natbib}

\begin{document}
\title{Diplomityövertailu} 
\subtitle{Teemu Partanen: Prerequisites for successful software maintenance and a reference model of the maintenenace processes}
\author{Antti Heikkonen} 
\date{\today}

\frame{\titlepage} 

\frame{\frametitle{Sisältö}\tableofcontents} 

\section{Työn taustat} 

\frame{\frametitle{Työn tarkoitus} 
Työssä arvioitiin ohjelmistojen huolto/ylläpitoprosesseja (\emph{software maintenenance processes}) Ixonos Technology Consulting Ltd:lle, joka tarjoaa sekä kehitys-, että ylläpitopalveluita asiakkailleen. \pause
\\[0.2in]
Miksi? % dipassa: maksaa (hallinta, kehitys), ei huomioitu tutkimuksessa tai organisaatioissa
\begin{itemize} 
\item Toistuva prosessi, jota kannattaa optimoida
\item Elinkaarinäkökulma: kustannuksia ja (lisä)myyntiä myös toimituksen jälkeen
\end{itemize}
}
% process issues: uniikit vaatimukset ja technical issues: miten käytännössä toteutetaan. viitekehys unohdetaan

% Kysymystö vaikuttaa lavealta ja laajalta. Toisaalta se mukailee työn rakennetta: kappaleet vastaavat työn kysymyksiin. Haiskahtaa jälkeenpäin säädetyltä.
\frame{\frametitle{Tutkimuskysymykset}
Tutkimuskysymykset: \pause
\begin{itemize} %
\item Mitkä ovat ohjelmistoylläpidon edellyttämät prosessit? \pause
\item Mitä olemassa olevia ylläpitomalleja on ja miten ne sopivat kohdeyritykseen? \pause
\item Mitä prosesseja kohdeyritys edellyttää ylläpitomallilta? \pause
\end{itemize}
Lisäksi
\begin{itemize}
\item Mitä ongelmia on nykyisessä ylläpitomallissa? \pause
\item Miten ylläpitoa voidaan tukea jo ohjelmiston kehitysvaiheessa ongelmien välttämiseksi? % yhdistää teoriaa ja empiriaa
\end{itemize}
}

% suurin ongelma että ei kuvata empiriistä toimenpidettä tarpeeksi tarkasti. mitä uutta luodaan ylläpitomallin tutkimisen ja referenssimallin luonnin ohella. dippa tuntuu enemmänkin selvitykseltä kuin tutkimukselta joka tuottaa uutta tietoa

\frame{\frametitle{Mutta!}
Työn tavoitetta ei kuvata.
}

\section{Tutkimuksen rakenne} 

\frame{\frametitle{Tutkimuksen rakenne}
\begin{figure}
  \includegraphics[width=6cm]{partanen_tutkimus.png}
    \caption{Partasen diplomityön rakenne}
    \label{fig:rakenne}
\end{figure}
}

\frame{\frametitle{Tutkimusmenetelmät}
Tutkimusmenetelminä action research ja konstruktiivinen tutkimus.
}

\section{Teoria} % Todella helppolukuista ja sujuvasti etenevää, mutta miksi juuri nämä asiat käsitellään.
\frame{\frametitle{Ohjelmistoylläpidon teoria}
Yleisten määritelmien ohella teoriaosuus esittelee
\begin{itemize}
\item Ylläpitoprosessien typologian, % menetelmät
\item Palvelunäkökulman ohjelmistoylläpitoon, % suhteet
\item Elinkaarinäkökulman ohjelmistoihin, % ympäristö
\item Neljä ylläpidon viitekehystä ja näiden vertailun, % luettele jos jotakuta kiinnostaa
\end{itemize}
joista kootaan \bf{referenssimalli}.
}
% referenssimallit käydään läpi aika tarkasti, mutta näiden valitsemiselle esitettiin kriteerit/hyvä scope. mallit kuvattu hyvin
% vertailussakin kriteeristö, joka esitellään, mutta vedetään vähän hatusta . plussaa oli että näitä käytettiin referenssimallin kehittämisessä eivätkä nämä jääneet näin irrallisiksi
% referenssimallin rakentamista olisi voinut korostaa alussa. tutkimusongelma jää edelleen epämääräiseksi selvittelyksi...
% referenssimalli käydään läpi todella tarkkaan prosessi prosessilta. tässä oma dippa eroaa aika paljon, vaikka karkea rakenne on sama
% rakenneharjoitus
% lopussa teksti vähän heikkenee (kirjoitettu viimeiseksi? dipan versio ei lopullinen, Huom!)

\frame{\frametitle{Ohjelmistoylläpidon teoria}
laita mallin kuva tähän
}

\section{Empiria}
\frame{\frametitle{Ylläpitomentelmät Ixonosilla}
Haastatteluihin perustuva selvitys pureutui:
\begin{itemize}
\item Ongelmakohtiin valituissa projekteissa,
\item Ylläpitäjien kokemuksiin,
\item Asiakasedustajan kokemuksiin
\end{itemize}
ja tiivisti tulokset parannusehdotuksiin ja \bf{ylläpidon tarkistuslistaan} \it{(maintainability checklist)}.
}
%latistavaa kuulla että ylläpito enimmäkseen inaktiivista
% neljännen tason väliostikoita eikä kunnon johdatusta sisältöön, ongelmakohdat vain listataan

\frame{\frametitle{Uuden mallin arviointi -workshop}

Referenssimalli esitettiin ylläpitäjille validointiworkshopissa. 
Tulokset:
\begin{itemize}
\item Ehdotuksia uusista prosesseista
\item Päivitetty refenssimalli % lisätty prosesseja joita ei aiemmin ollut
\item Referenssimallista johdettu asiakaslähetöinen malli
\end{itemize}
}

\section{Yhteenveto}

\frame{\frametitle{Tutkimuksen arviointi}
Tutkimus täyttää hyvän tutkimuksen kriteerit, mutta referenssimalli kehitetty kohdeyrityksen tarpeisiin, eikä sitä ole sovellettu ja arvioitu päivittäisessä työssä.
}

\frame{\frametitle{Kokonaisvaikutelma Partasen dipasta}
\begin{itemize}
Hyvää
\item Nopeasti asiaan.
\item Selkeä perusrakenne ja jouheva eteneminen: teoriatutkimus, synteesi, koe, johtopäätökset.
\item Perusteellinen ja analyyttinen käsittely
Huonoa
\item Epäselvät tavoitteet
\item Epäselvä johdattelu empiriaan ja to
\end{itemize}
}

\end{document}